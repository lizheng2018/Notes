\documentclass[UTF8]{ctexart}

% 一些基本包,类似于机器学习导入第三方库,作用见网上
\usepackage{fontspec}
\usepackage{lipsum} 
\usepackage{graphicx}
\usepackage{multirow}
\usepackage{booktabs}
\usepackage{geometry}
\usepackage{hyperref}

% 设置页边距,符合IEEE 的要求
\geometry{a4paper,left=1.65cm,right=1.65cm,top=1.78cm,bottom=1.78cm}

% 设置预览字体
\setmainfont[Mapping=tex-text]{Microsoft YaHei}

\title{标题}
\date{ }  % 空白表示不显示日期
\begin{document}
\maketitle
%\tableofcontents % 插入目录 

此模板是一个基础的latex写作结构,其它更加高级的用法还需慢慢探索

\section{一级标题}

可以直接在这些空白行填写内容了

\subsection{二级标题:随机生成文本}

\lipsum[1-2] % 随机生成一些文本,[1-2]代表只生成2段,默认为10段

\subsubsection{文本常用格式}

意大利斜体: \textit{italic}

加粗命令: \textbf{文本}\\ %双斜杠代表后续空一行

我们过了江,进了车站。我买票,他忙着照看行李。行李太多,得向脚夫行些小费才可过去。他便又忙着和他们讲价钱。我那时真是聪明过分,总觉他说话不大漂亮,非自己插嘴不可,但他终于讲定了价钱;就送我上车。他给我拣定了靠车门的一张椅子;我将他给我做的紫毛大衣铺好座位。他嘱我路上小心,夜里要警醒些,不要受凉。又嘱托茶房好好照应我。我心里暗笑他的迂;他们只认得钱,托他们只是白托!而且我这样大年纪的人,难道还不能料理自己么?我现在想想,我那时真是太聪明了。

\subsubsection{公式表示}

A word of warning: $R^{i}_{j}=\sqrt{k}$ Wrapping figures in LaTeX will require a lot of manual adjustment of your document. There are several packages available for the task, but none of them works perfectly. Before you make the choice of including figures with text wrapping in your document Figure~\ref{好看的图片}, make sure you have considered all the options. For example, you Equa~\ref{eqa:xli} could use a layout with two columns for your documents and have no text-wrapping at all.

写公式,这种写公式的格式于上面两个 $  $ 美元符号是不同的,下面是行间模式,上面是行内模式,具体效果自行预览查看,公式中各种符号的使用,请查看相关文档,网上很多人有总结,比如\url{https://www.jianshu.com/p/0d442af1541e},\url{https://liam.page/2014/09/08/latex-introduction},可以结合texstudio的相关快捷符号快速使用

ps:上面url链接引用方式需要先导入包\textbf{hyperref}

\begin{equation}
	\label{eqa:xli} % 公式名字
	V_{outliers}=\{v_i \mid v_i>(\mu +3\sigma) \land v_i<(\mu -\frac{\mu}{3})\}
\end{equation}



\subsection{二级标题:列表表示}

小草偷偷地从土里钻出来,嫩嫩的,绿绿的。Table~\ref{tab:li}园子里,田野里,瞧去,一大片一大片满是的。坐着,躺着,打两个滚,踢几脚球,赛几趟跑,捉几回迷藏。风轻悄悄的,草软绵绵的。

桃树、杏树、梨树,你不让我,我不让你,都开满了花赶趟儿。红的像火,粉的像霞,白的像雪。花里带着甜味儿;闭了眼,树上仿佛已经满是桃儿、杏儿、梨儿。花下成千成百的蜜蜂嗡嗡地闹着,Figure~\ref{fig:modell} 大小的蝴蝶飞来飞去。野花遍地是:杂样儿,有名字的,没名字的,散在草丛里,像眼睛,像星星,还眨呀眨的。


以下是列表表达方法,itemize 以圆点作为标签,enumerate 是有序的列表,这两种常用
\begin{itemize}
	\item 第一个
	\item 第二个
	\item 第三个
\end{itemize}



\begin{enumerate}
	\item 第一个
	\item 第二个
	\item 第三个
\end{enumerate}


\section{一级标题:示例}

“吹面不寒杨柳风”,不错的,像母亲的手抚摸着你。风里带来些新翻的泥土的气息,混着青草味儿,还有各种花的香,都在微微润湿的空气里酝酿。鸟儿将巢安在繁花嫩叶当中,高兴起来了,呼朋引伴地卖弄清脆的喉咙,唱出宛转的曲子,与轻风流水应和着。牛背上牧童的短笛,这时候也成天嘹亮地响着。

雨是最寻常的,一下就是三两天。可别恼。看,像牛毛,像花针,像细丝,密密地斜织着,人家屋顶上全笼着一层薄烟。树叶儿却绿得发亮,小草儿也青得逼你的眼。傍晚时候,上灯了,一点点黄晕的光,烘托出一片安静而和平的夜。在乡下,小路上,石桥边,有撑起伞慢慢走着的人,地里还有工作的农民,披着蓑戴着笠。他们的房屋,稀稀疏疏的在雨里静默着。

天上风筝渐渐多了,地上孩子也多了。城里乡下,家家户户,老老小小,也赶趟儿似的,一个个都出来了。舒活舒活筋骨,抖擞抖擞精神,各做各的一份事去。“一年之计在于春”,刚起头儿,有的是工夫,有的是希望。

春天像刚落地的娃娃,从头到脚都是新的,它生长着。

春天像小姑娘,花枝招展的,笑着,走着。

春天像健壮的青年,有铁一般的胳膊和腰脚,领着我们上前去。

\section{插入图片和表格}

插入图片和表格

插入图片:latex会按你的图片插入位置来默认计数,从图1开始
\begin{figure}
	\centering
	\includegraphics[scale=.75]{figs/1.jpg} % 前者是图片缩小倍数,有时候图片很大,需要放缩,后者是同一文件夹下图片路径
	\caption{很好看的图片,是吧} % 图片描述
	\label{好看的图片} % 图片名称,在文章中引用时直接输入名称即可,比如下面一段话
\end{figure}

春天像刚落地的娃娃,从头到脚都是新的,它生长着Figure~\ref{好看的图片}。

\begin{figure}
	\centering
	\includegraphics[scale=.75]{figs/1.jpg}
	\caption{The model structure of the DMF network.}
	\label{fig:modell}
\end{figure}

表格插入方法,见\url{https://www.cnblogs.com/tsingke/p/6106733.html},很详细,Excel2LaTeX.xla宏文件已提供

% Table generated by Excel2LaTeX from sheet 'Sheet1'
\begin{table}[htbp]
	\centering
	\caption{这是一个表格} % 表格名称
	\begin{tabular}{lrrr}
		\toprule
		City  & \multicolumn{1}{l}{Longitude} & \multicolumn{2}{c}{Latitude} \\
		\midrule
		\multicolumn{1}{c}{\multirow{3}[6]{*}{蚌埠}} & 117.3961 & 32.939 & 38.1035 \\
		\cmidrule{2-4}          & 117.3575 & 32.9444 & 40.89986 \\
		\cmidrule{2-4}          & 117.3086 & 32.935 & 43.22555 \\
		\midrule
		蚌埠    & 117.4186 & 32.8913 & 39.13876 \\
		\midrule
		蚌埠    & 117.3536 & 32.9673 & 39.14484 \\
		\midrule
		蚌埠    & 117.3065 & 32.8985 & 39.61144 \\
		\midrule
		淮南    & 117.0417 & 32.6611 & 35.59891 \\
		\midrule
		淮南    & 116.8028 & 32.7639 & 32.52687 \\
		\midrule
		淮南    & 117.0083 & 32.6450 & 40.44374 \\
		\midrule
		淮南    & 116.8556 & 32.6028 & 65.59483 \\
		\bottomrule
	\end{tabular}%
	\label{tab:li}%
\end{table}%


\section{结束}

简单介绍,到此结束,不过,踩坑才刚刚开始!

\end{document}